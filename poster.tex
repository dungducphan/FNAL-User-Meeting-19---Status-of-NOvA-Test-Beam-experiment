\documentclass[a0paper,portrait]{baposter}


\usepackage{wrapfig}
\usepackage{lmodern}
\usepackage[demo]{graphicx}
\usepackage{caption}
\usepackage{subcaption}
\usepackage{graphicx}

\usepackage[utf8]{inputenc} %unicode support
\usepackage[T1]{fontenc}


\selectcolormodel{cmyk}

\graphicspath{{figures/}} % Directory in which figures are stored


\newcommand{\compresslist}{%
\setlength{\itemsep}{0pt}%
\setlength{\parskip}{1pt}%
\setlength{\parsep}{0pt}%
}

\newenvironment{boenumerate}
  {\begin{enumerate}\renewcommand\labelenumi{\textbf\theenumi.}}
  {\end{enumerate}}



\begin{document}


\definecolor{darkgreen}{cmyk}{0.8,0,0.8,0.45}
\definecolor{lightgreen}{cmyk}{0.8,0,0.8,0.25}

\begin{poster}
{
grid=false,
headerborder=open, % Adds a border around the header of content boxes
colspacing=1em, % Column spacing
bgColorOne=white, % Background color for the gradient on the left side of the poster
bgColorTwo=white, % Background color for the gradient on the right side of the poster
borderColor=darkgreen, % Border color
headerColorOne=lightgreen, % Background color for the header in the content boxes (left side)
headerColorTwo=lightgreen, % Background color for the header in the content boxes (right side)
headerFontColor=white, % Text color for the header text in the content boxes
boxColorOne=white, % Background color of the content boxes
textborder=rounded, %rectangle, % Format of the border around content boxes, can be: none, bars, coils, triangles, rectangle, rounded, roundedsmall, roundedright or faded
eyecatcher=false, % Set to false for ignoring the left logo in the title and move the title left
headerheight=0.08\textheight, % Height of the header
headershape=rounded, % Specify the rounded corner in the content box headers, can be: rectangle, small-rounded, roundedright, roundedleft or rounded
headershade=plain,
headerfont=\Large\textsf, % Large, bold and sans serif font in the headers of content boxes
%textfont={\setlength{\parindent}{1.5em}}, % Uncomment for paragraph indentation
linewidth=2pt % Width of the border lines around content boxes
}
{}
%
%----------------------------------------------------------------------------------------
%	TITLE AND AUTHOR NAME
%----------------------------------------------------------------------------------------
%
{\textsf %Sans Serif
{\color{green} The NOvA Test Beam Experiment}
} % Poster title
% {\vspace{1em} Marta Stepniewska, Pawel Siedlecki\\ % Author names
% {\small \vspace{0.7em} Department of Bioinformatics, Institute of Biochemistry and Biophysics, PAS, Warsaw, Pawinskiego 5a}} % Author email addresses
{\sf\\
\underline{Dung Phan} and Beatriz Tapia Oregui (On behalf of the NOvA collaboration)
\vspace{0.1em}\\
\small{University of Texas at Austin}
}
{\includegraphics[scale=0.35]{logo}} % University/lab logo

\headerbox{Motivation of NOvA Test Beam}
{name=introduction,column=0,row=0, span=2}{
\begin{itemize}
    \item NOvA is an off-axis long-baseline accelerator neutrino oscillation experiment.
    \item Main physics goals: measurements of $\nu_\mu$($\bar{\nu}_\mu$) disappearance and $\nu_e$($\bar{\nu}_e$) appearance, precision measurement of $\theta_{23}$, probing neutrino mass ordering and the CP violating phase $\delta_{\text{CP}}$.
    \item The NOvA Test Beam will assist NOvA in reaching these goals by studying the limiting factors of current analyses.
\end{itemize}

\begin{center}
    \includegraphics[width=0.4\columnwidth]{NOvA/sin.pdf}
    \includegraphics[width=0.4\columnwidth]{NOvA/dm.pdf}
    
    Dominant systematic errors of the measurement of $\sin^2\theta_{23}$ and $\Delta m^2_{32}$. Highlighted in green are those that NOvA Test Beam will be able to improve.
\end{center}
}


\headerbox{MC7 Beamline}{name=mcs,column=0,below=introduction,span=2}{

\begin{itemize}
    \item Located in the MC7b enclosure at the FTBF. 
    \item Secondary beam of 64~GeV$/c$ protons hits a copper target: produces tertiary beam composed primarily of $p$, $\pi$ and a small contribution from $e$, $\mu$ and $K$.
    \item Beamline components (right): 2 scintillator paddles for time-of-flight measurements, 4 wire chambers for momentum measurements and a Cherenkov counter to tag electrons.
    \item Equipped with two collimators and a dipole magnet, which guides particles within a momentum range from 0.3 to 2~GeV$/c$. Momenta and counts of particles entering the detector after traversing all beamline components (left).
\end{itemize}
\begin{center}
    \includegraphics[height=4.8cm]{NOvA/part.png}
    \includegraphics[height=4cm]{NOvA/diagram.png}
\end{center}
}

\headerbox{NOvA Test Beam Detector}{name=screen,span=1,column=2}{ 
\begin{itemize}
\item Similar design to NOvA's Near and Far detectors, but smaller in size: 63 planes of plastic extrusion modules filled with liquid scintillator.
\item Size was chosen based on containment study determining the range of different particles at different momenta. 
\item Utilizes both FD and ND front-end boards to study electronic response differences between them. So far, 32 out of 63 planes have been filled with liquid scintillator and hooked up to front-end electronics. \\
\end{itemize}
\begin{center}
    \includegraphics[width=0.73\linewidth]{NOvA/sizes.pdf}
\end{center}
}

\headerbox{Detector commissioning}{name=references,column=2,span=1,below=screen}{
\begin{itemize}
    \item Commissioning runs in progress, cosmic and beam data taking. 
    \item Currently focusing on synchronizing with beamline components.
\end{itemize}

\begin{center}
  \includegraphics[width=0.95\linewidth]{NOvA/display.pdf}  
  Display of a cosmic ray event.
\end{center}
}


\headerbox{Beamline Commissioning}{name=sea,span=2,column=0,below=mcs}{
A set of reconstruction algorithms is in place. Tracks of beam particles are reconstructed from MWPC, allowing momentum measurement (left). Together with Time-of-Flight (right) and electron selection from the Cherenkov counter, the PID of beam particles can be performed.

\begin{center}
    \includegraphics[height=3.8cm]{prem/momentum_dist.png}
    \hspace{1cm}
    \includegraphics[height=3.8cm]{prem/tof.png}
    
\end{center}
}

\headerbox{Status and plans}{name=conclusion,column=2,below=references,span=1}{

\begin{itemize}
    \item Beamline detectors' instrumentation and installation has been finalized. Timing calibration is in progress.
    \item Remaining 31 planes of the second block will be filled during the 2019 shutdown.
    \item Status of all beamline components and detector can be monitored and controlled via Synoptic system. 
    \item Reconstruction are developed, tested and used in beamline commissioning to study momentum, TOF and PID.
\end{itemize}
}

\headerbox{Come visit!}{name=sea,column=0,below=sea,span=3,above=bottom}{
\begin{center}
    \includegraphics[height=3.2cm]{NOvA/testbeam.png}
    \includegraphics[height=3.2cm]{NOvA/map.png}
\end{center}
}

\end{poster}

\end{document}
